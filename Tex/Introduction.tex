\section{背景}
近年来,深度学习在许多机器学习任务上取得了革命性的进展,比如图像分类、视频处理、语音识别、自然语言理解等。深度学习在这些领域的成功部分归因于快速发展的计算资源(如GPU),大量训练数据的可用性,以及深度学习从欧几里得数据抽取特征表示的有效性。图像、文本和视频等均可表示为欧几里得空间中的数据。以图像为例,我们可以将图像表示为欧几里得空间中的规则网络,卷积神经网络(GNN)\cite{NIPS2012_4824}能够利用图像数据的平移不变形,局部连通性和合成性,从而提取与整个数据集共享的局部特征,以进行各种图像分析。

与此同时,图(Graph)数据易于描述对象之间的复杂关系,因此越来越来的任务用图结构来表示数据,比如社交网络、推荐系统以及知识图谱等。以社交网络为例,在社交网络图中,节点表示个人或组织,而边表示节点之间的各种关系,通过对社交网络进行挖掘,能够发现虚拟社区,进行用户行为分析等。然而,图数据的复杂性对现有的机器学习算法提出了重大挑战。由于图具有复杂的拓扑结构,没有固定的节点顺序,并且图可能是动态的,因此一些重要的操作例如卷积,在图像域中易于计算,但是应用于图域却非常困难。此外,现有机器学习算法的核心假设是实例彼此独立,但是该假设不再适用于图数据,因为每个实例(节点)通过引用、朋友关系等各种类型的链接与其它实例相关联。

为了应对图数据问题的复杂性,图表示学习应运而生。
