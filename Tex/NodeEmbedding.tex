\UseRawInputEncoding
\documentclass{article}
\usepackage{ctex}
\usepackage{color}
\usepackage{graphicx}
%\usepackage{multicol}
\usepackage{CJK}

\begin{document}

\section{Node embedding}

\subsection{LINE}

\subsection{DeepWalk}

\subsection{Node2vec}
在网络中的节点和边缘上的预测任务需要在学习算法所使用的工程特征方面花费精力。最近在更广泛的表征学习领域的研究通过学习特征本身在自动化预测方面取得了重大进展。在Node2vec中,学习了节点到特征的低维空间的映射,这最大化了保留节点的网络邻域的可能性。定义了节点网络邻域的灵活概念,并设计了一个有偏差的随机游走过程,有效地探索了不同的邻域。该算法推广了基于网络邻域的严格概念的先前工作,并且作者认为探索邻域的额外灵活性是学习更丰富表示的关键。


\end{document}

